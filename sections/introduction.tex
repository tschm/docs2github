\section{Introduction}

When Harry Markowitz \cite{markowitz1952portfolio} published
his seminal paper in 1952 he was a 25-year-old graduate student at
the University of Chicago.
Little did he know that his work would become the foundation of
modern portfolio theory, and methods inspired by his work
would become the most widely used portfolio construction method
of the next 70 years (and counting).
In the same year Hestenes and Stiefel introduced
the method of conjugate gradients \cite{hestenes1952} which
also started life as an academic exercise without any immediate impact.

For his initial experiments he created a programming
language and invented the critical line algorithm \cite{markowitz1955cla}.
The lack of data and accessible computational power \cite{markowitz2019aqr}
rendered his ideas impractical for many years despite his pragmatic approach.

Markowitz has been a true pioneer for a field
we call today computational finance.
His language is long gone in finance\footnote{Apparently the US military
is still using a language based on his ideas, see \cite{markowitz2019aqr} for a
history of \emph{SIMSCRIPT}.} but his core ideas
survived and remain at the epicentre of
professional quantitative finance.

In 1963, Sharpe \cite{sharpe1963} published his market model,
designed to speed up the Markowitz calculations.
His model was a one-factor risk model (the market),
with the assumption that all residual returns are
uncorrelated.
His paper stated that solving a 100-asset
problem on an IBM 7090 computer required 33 minutes,
but his simplified risk model reduced that to 30 seconds.
He also commented that that computer could only handle 249
assets at most with a full covariance matrix, but 2,000
assets with the simplified risk model.

Numerous contributions and research areas branched off from those
developments, including the capital asset pricing model (CAPM),
the arbitrage pricing theory (APT), the Fama-French three-factor model,
the Carhart four-factor model, and many others.
The common theme of all those models is that they are
based on the Markowitz framework.

In 1990, Markowitz was awarded the Nobel Prize in Economics
for his work on portfolio theory.
He shared the prize with Merton Miller and William Sharpe.

Industry took a while to adopt his ideas.
An entire industry of portfolio managers had to learn about linear algebra, convex optimization, and
computational finance.
Markowitz's ideas changed the face of a professional community
and the way we think about investing.

The constant attacks on Markowitz's work are a testament to its importance.
Those attacks usually fit into one or multiple of the following categories:

\BIT
\item It appears to not be based on a concave utility function.
If you make too much,
that's really bad, according to the Markowitz objective.
(But it \emph{is}, in fact,
a maximum utility method.)  This the embarrassing upside risk problem.
Markowitz doesn't want you to make too much.

xxx:
Markowitz is at its core a trade-off between risk and return.
Typically, a maximum utility method is used to balance
the two objectives.
One route of attack is to argue that Markowitz
prevents you from making too much money by
keeping the upside risk in check.

\item
It only looks at the second moment of the portfolio return.
But we really care about downside risk, the left tail, \etc.


\item
It can be famously sensitive to the input data \cite{michaud1998efficient}.
With inappropriate or naive choice of
data, you get silly portfolios, say, concentrated in a few names or even just one.
If you give it a not-full-rank covariance, watch out!

\item Markowitz is a greedy method.
It only looks head one step, whereas we'd prefer a real
stochastic policy, that understands it will have recourse.
\EIT

These are true, on their face.
But when we dive deeper we see that there are very good responses
to these so-called limitations.
In this paper we would like to share our
recommendations to make Markowitz work in practice.
Despite its age, and the many attacks on it,
it is still used everywhere.

Plugging in estimated values of expected data into a textbook Markowitz formulation
is a recipe for disaster.
It is doomed to fail.
%And yet, it is surprising how many try exactly that.

This might be a problem of the finance community at large.
Other groups also develop policies or models based on optimization.
No one just reads the textbook version of linear quadratic control
and makes it work in practice.
And yet, engineers manage control systems performing
robustly and safely.
Every airplane relies on their expertise.
Engineers know a bunch of tricks and recipes.
Our goal is to give a short list of the analogous tricks and tips for Markowitz.

In \cite{schmelzer2013seven} we gave a more
negative outlook on issues
we have seen in practice.
Here we follow a more optimistic and constructive path.
We are inspired by the Numerical Recipes \cite{press1992numerical}.
We show how to mitigate most of the aforementioned issues and
using code fragments we implement a competitive Markowitz solver.

We don't cover theory, or statistics, although we will make occasional references to
them, when it suits us.
We take convex optimization as simply a technology, that can and should be used
to construct portfolios.
We take its reliability and speed as technological facts.

%\paragraph{Convex optimization.}
%Matters for reliability, backtest speed, etc.
%We recommend revisiting the original paper as a first step.
%Markowitz did not suggest mean-variance optimization! He understood his paper
%as the 2nd step of a process relying on estimates for the expected return and the covariance matrix in the first step.
%He was proposing a rather quantamental approach

%Original paper 1952, more than 57000 citations
%The Sharpe ratio wasn't invented yet, Markowitz calls it the E-V rule (policy
%or portfolio) Very much in favor of a quantamental approach (see last page).
%Markowitz understand his paper as step 2 and in the process from the believes
%to the potfolio and argued that arriving at those believes is step 1.

%Practioners struggle to render their believes and often apply all sorts of
%steps to move the solution closer to a desired final result. Portfolio
%construction shouldn't be understood as a tool to confirm your personal bias.

%In this paper we show how to mitigate most of the aforementioned problems and
%using code fragments we show how to implement a competitive Markowitz solver in
%2023.

%We have observed and written about frequent problems (see Schmelzer) but here
%we want to share a more optimistic and constructive approach giving an update
%without too much mathematical detail and rigour.

Most of the material in this paper is not new but scattered across many sources.
We believe that it is useful to have a self-contained reference addressing essentially
all practical issues we are aware of in one paper.

Not everything we do is widely accepted as industry standard.
Two points, in particular, are rarely discussed in literature
and even less commonly practiced outside of academia.
We believe that our strategy to address hard constraints
and the use of uncertainty for both the expected returns and the expected covariance
is not yet main stream.

We have created a companion software package we have used for all experiments in
this paper.
It is available on github \cite{cvxmarkowitz}.
