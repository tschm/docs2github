\documentclass[12pt]{article}
\usepackage{fullpage,graphicx,psfrag,amsmath,amsfonts,verbatim}
\usepackage[small,bf]{caption}
\usepackage{amssymb}

\input{defs}

% \usepackage[round]{natbib}
\bibliographystyle{alpha}

\usepackage{mathtools}
\usepackage{adjustbox}
\usepackage{booktabs}

\usepackage{hyperref}
\hypersetup{ colorlinks   = true,    % Colours links instead of ugly boxes
urlcolor     = blue,    % Colour for external hyperlinks
linkcolor    = blue,    % Colour of internal links
citecolor    = red      % Colour of citations
}

\title{Making Markowitz Work\\[1ex] (Working Title)}
\author{Stephen Boyd \footnote{Alphabetical order.}
\and Kasper Johansson 
\and Ronald Kahn
\and Philipp Schiele
\and Thomas Schmelzer
}

\begin{document}
\maketitle

\begin{abstract}
Ja, Wir werden einen brauchen.  Eventually.
\end{abstract}

\clearpage
\tableofcontents
\clearpage

\section{Introduction}

When Harry Markowitz \cite{markowitz1952portfolio} published
his seminal paper in 1952 he was a 25-year-old graduate student at
the University of Chicago.
Little did he know that his work would become the foundation of
modern portfolio theory, and methods inspired by his work
would become the most widely used portfolio construction method
of the next 70 years (and counting).
In the same year Hestenes and Stiefel introduced
the method of conjugate gradients \cite{hestenes1952} which
also started life as an academic exercise without any immediate impact.

For his initial experiments he created a programming
language and invented the critical line algorithm \cite{markowitz1955cla}.
The lack of data and accessible computational power \cite{markowitz2019aqr}
rendered his ideas impractical for many years despite his pragmatic approach.

Markowitz has been a true pioneer for a field
we call today computational finance.
His language is long gone in finance\footnote{Apparently the US military
is still using a language based on his ideas, see \cite{markowitz2019aqr} for a
history of \emph{SIMSCRIPT}.} but his core ideas
survived and remain at the epicentre of
professional quantitative finance.

In 1963, Sharpe \cite{sharpe1963} published his market model,
designed to speed up the Markowitz calculations.
His model was a one-factor risk model (the market),
with the assumption that all residual returns are
uncorrelated.
His paper stated that solving a 100-asset
problem on an IBM 7090 computer required 33 minutes,
but his simplified risk model reduced that to 30 seconds.
He also commented that that computer could only handle 249
assets at most with a full covariance matrix, but 2,000
assets with the simplified risk model.

Numerous contributions and research areas branched off from those
developments, including the capital asset pricing model (CAPM),
the arbitrage pricing theory (APT), the Fama-French three-factor model,
the Carhart four-factor model, and many others.
The common theme of all those models is that they are
based on the Markowitz framework.

In 1990, Markowitz was awarded the Nobel Prize in Economics
for his work on portfolio theory.
He shared the prize with Merton Miller and William Sharpe.

Industry took a while to adopt his ideas.
An entire industry of portfolio managers had to learn about linear algebra, convex optimization, and
computational finance.
Markowitz's ideas changed the face of a professional community
and the way we think about investing.

The constant attacks on Markowitz's work are a testament to its importance.
Those attacks usually fit into one or multiple of the following categories:

\BIT
\item It appears to not be based on a concave utility function.
If you make too much,
that's really bad, according to the Markowitz objective.
(But it \emph{is}, in fact,
a maximum utility method.)  This the embarrassing upside risk problem.
Markowitz doesn't want you to make too much.

xxx:
Markowitz is at its core a trade-off between risk and return.
Typically, a maximum utility method is used to balance
the two objectives.
One route of attack is to argue that Markowitz
prevents you from making too much money by
keeping the upside risk in check.

\item
It only looks at the second moment of the portfolio return.
But we really care about downside risk, the left tail, \etc.


\item
It can be famously sensitive to the input data \cite{michaud1998efficient}.
With inappropriate or naive choice of
data, you get silly portfolios, say, concentrated in a few names or even just one.
If you give it a not-full-rank covariance, watch out!

\item Markowitz is a greedy method.
It only looks head one step, whereas we'd prefer a real
stochastic policy, that understands it will have recourse.
\EIT

These are true, on their face.
But when we dive deeper we see that there are very good responses
to these so-called limitations.
In this paper we would like to share our
recommendations to make Markowitz work in practice.
Despite its age, and the many attacks on it,
it is still used everywhere.

Plugging in estimated values of expected data into a textbook Markowitz formulation
is a recipe for disaster.
It is doomed to fail.
%And yet, it is surprising how many try exactly that.

This might be a problem of the finance community at large.
Other groups also develop policies or models based on optimization.
No one just reads the textbook version of linear quadratic control
and makes it work in practice.
And yet, engineers manage control systems performing
robustly and safely.
Every airplane relies on their expertise.
Engineers know a bunch of tricks and recipes.
Our goal is to give a short list of the analogous tricks and tips for Markowitz.

In \cite{schmelzer2013seven} we gave a more
negative outlook on issues
we have seen in practice.
Here we follow a more optimistic and constructive path.
We are inspired by the Numerical Recipes \cite{press1992numerical}.
We show how to mitigate most of the aforementioned issues and
using code fragments we implement a competitive Markowitz solver.

We don't cover theory, or statistics, although we will make occasional references to
them, when it suits us.
We take convex optimization as simply a technology, that can and should be used
to construct portfolios.
We take its reliability and speed as technological facts.

%\paragraph{Convex optimization.}
%Matters for reliability, backtest speed, etc.
%We recommend revisiting the original paper as a first step.
%Markowitz did not suggest mean-variance optimization! He understood his paper
%as the 2nd step of a process relying on estimates for the expected return and the covariance matrix in the first step.
%He was proposing a rather quantamental approach

%Original paper 1952, more than 57000 citations
%The Sharpe ratio wasn't invented yet, Markowitz calls it the E-V rule (policy
%or portfolio) Very much in favor of a quantamental approach (see last page).
%Markowitz understand his paper as step 2 and in the process from the believes
%to the potfolio and argued that arriving at those believes is step 1.

%Practioners struggle to render their believes and often apply all sorts of
%steps to move the solution closer to a desired final result. Portfolio
%construction shouldn't be understood as a tool to confirm your personal bias.

%In this paper we show how to mitigate most of the aforementioned problems and
%using code fragments we show how to implement a competitive Markowitz solver in
%2023.

%We have observed and written about frequent problems (see Schmelzer) but here
%we want to share a more optimistic and constructive approach giving an update
%without too much mathematical detail and rigour.

Most of the material in this paper is not new but scattered across many sources.
We believe that it is useful to have a self-contained reference addressing essentially
all practical issues we are aware of in one paper.

Not everything we do is widely accepted as industry standard.
Two points, in particular, are rarely discussed in literature
and even less commonly practiced outside of academia.
We believe that our strategy to address hard constraints
and the use of uncertainty for both the expected returns and the expected covariance
is not yet main stream.

We have created a companion software package we have used for all experiments in
this paper.
It is available on github \cite{cvxmarkowitz}.


\section{The original Markowitz problem} XXX Kasper: rewrite per Thomas's
feedback XXX
Markowitz~\cite{markowitz1952portfolio} introduced his, now famous, portfolio
optimization formulation as step two in a two-step process. The first step is
devoted to the estimation of the expected value $\mu \in \reals^n$ of the
returns of $n$ assets and the covariance matrix $\Sigma \in \reals^{n\times n}$ of
these returns. The second step assumes that an investor has a desire to
maximize the expected return of the portfolio while keeping the risk at a
minimum. Given $\mu$ and $\Sigma$ Markowitz proposed to solve the following
optimization problem to find the optimal portfolio weights $w \in
\reals^n$\footnote{In his first analysis of the problem, Markowitz for
simplicity and illustrative purposes also assumed a
long-only constraint, \ie, $w \geq 0$.}:
\BEQ \label{eq:original-markowitz}
\begin{array}{ll}
\mbox{maximize}   &  w^T\mu - \lambda w^T \Sigma w\\
\mbox{subject to} &  \ones^T w = 1,\\
\end{array}  
\EEQ
with variable $w$, where $\lambda$ is a risk aversion parameter that controls
the trade-off between risk and return. Here, $ w^T\mu$ denotes the expected
return of the portfolio, and $w^T \Sigma w$ its variance. There are several equivalent
formulations of this problem. For example, we could maximize the expected
return, subject to a maximum variance constraint, $\lambda w^T \Sigma w \leq
\sigma_{\max}^2$, or we could minimize the variance, subject to a minimum return
constraint, $w^T\mu \geq \mu_{\min}$. For certain choices of $\lambda,
\sigma_{\max}$, and $\mu_{\min}$, these problems are all
equivalent~\cite{boydconvex}. Problem~\eqref{eq:original-markowitz} has the
analytical solution 
\[
w^{\star} = \frac{1}{2\lambda} \Sigma^{-1} (\mu+\nu^{\star}\ones),
\]
where
\[
\nu^{\star} = \frac{2\lambda - \ones^T \Sigma^{-1}\mu}{\ones^T \Sigma^{-1} \ones}
\]
is the optimal dual variable~\cite{boydconvex}.

\subsection{Extensions} 
The original Markowitz problem~\eqref{eq:original-markowitz} allows the investor
to maximize her expected return for a certain level of risk. Convex optimization
allows for many extensions of this problem, like adding constraint on, or penalizing, positions,
leverage, turnover, etc.

\subsection{Criticisms}
Markowitz has been heavily criticized for many reasons. Firstly, it has been
argued to be an estimation-error maximization problem, meaning that it
overemphasizes assets with high estimation errors. It has also been argued that
it relies on sample estimates of the mean and covariance matrix, which are
suboptimal~\cite{michaud1989markowitz_enigma}. (Although this second criticism
relates to the first step of the two-step process, and not to the part that
Markowitz actually tried to solve.) Another criticism is that the optimizer does
not take into account differences in uncertainties of the mean
and covariance estimates of different assets. For example, an estimate of the return of an
equity return may be much more uncertain than the return estimate of a growth
factor. Other critics argue that mean-variance
optimizers can be unstable, that solutions may be non-unique, or that XXX look
into more sources. This is mostly based on Michaud's paper from 1989; a lot has
changed XXX XXX Target missing factors and non-financial structure XXX 

Although some of these criticisms are valid when taken at face value, this paper
will illustrate how to avoid them completely in practice, when doing Markowitz
right. 



\section{The Markowitz problem}

\subsection{Notation}

Our notation is close to the notation in \cite{BoydKahnMultiPeriod},
with the exception that we use separate notation for the non-cash and cash investments.
We also drop the subscript $t$ which denotes time period, and 
use it only when we are discussing simulation and backtesting.

\paragraph{Portfolio weights.}
We consider a portfolio consisting of investments (possibly long and short)
in $n$ assets, plus a cash account.
To describe the portfolio, we will work with the weights or fractions
of the total portfolio value for each asset. We denote
the weights for the assets as $w_i$, $i=1, \ldots, n$, and collect them 
into a portfolio weight vector $w =(w_1, \ldots, w_n)\in \reals^n$.
We denote the weight for the cash account as $c$, and the total
portfolio value (assumed positive) as $V$.
We interpret $w_i$ (or $c$) as the fraction of the total portfolio value $V$
held in asset $i$ (or cash).
The entries of $w$ can be negative, indicating a short position.
If $c$ is negative, it represents a loan.
The dollar value of asset $i$ held is $Vw_i$, and the dollar value of 
the cash account is $Vc$.
By definition we have 
\[
\ones^T w + c = 1,
\]
where $\ones$ is the vector
with all entries one. That is, the asset weights plus the cash weight sum to one.
We can express the cash weight in terms of the asset weights as $c=1-\ones^T w$.

We mention a few important concepts related to the weights.
A \emph{long-only} portfolio corresponds to all entries of $w$ being nonnegative,
which we write as $w \geq 0$ (which means elementwise).

The $130-30$ portfolio used to be a popular portfolio, see \cite{leibowitz2009modern}.
Here we sell $0.3 \times V$ by entering a short position to finance a $1.3 \times V$ long position.
This corresponds to the constraints $w^T \ones = 1.0$ but $\sum_{i=1}^n |w_i| = 1.6$.
We define the latter term as the \emph{leverage} of the portfolio, and denote it by $L$
\[
L = \sum_{i=1}^n |w_i| = \|w\|_1.
\]
(Several other closely related definitions are also used. Our definition is
commonly referred to as the \emph{gross leverage} \cite{ANG2011102}.)

Constraints on the leverage will be helpful to tame \emph{cash-neutral} portfolios
where $w^T \ones = 0$, $c=1$ and $L > 0$. Here the size of the long position is equal
to the size of the short position.

\paragraph{Gross portfolio return.}
We let $r_i$ denote the return of asset $i$ over the investment period.
We collect these asset returns into a return vector 
$r=(r_1, \ldots, r_n)\in \reals^n$.
These returns are adjusted for dividends, splits, and other corporate 
actions.
The portfolio return from asset $i$ is $r_i w_i$.
We let $r^\mathrm{rf}$ denote the risk-free interest rate, so the return in
the cash account is $r^\mathrm{rf}c$.
The (gross) total portfolio return is then 
\[
R = r^T w + r^\mathrm{rf} c.
\]
This gross return does not include costs such as borrowing cost (for assets or 
cash) or transaction costs, described below.

\paragraph{Benchmark.}
In some cases our focus is on portfolio performance 
relative to a benchmark portfolio.  We let $w^\mathrm{b}$ denote the weights of
the benchmark.  Typically the benchmark does not include any cash weight, \ie,
the benchmark cash weight is zero, so $\ones^T w^\mathrm{b} = 1$.
We refer to $w-w^\mathrm{b}$ as the \emph{active} weights of our portfolio.
The \emph{active portfolio return}, \ie, its return relative the benchmark, is 
\[
r^T w + r^\mathrm{rf} c - r^T w^\mathrm{b} = r^T (w-w^\mathrm{b}) + r^\mathrm{rf} c.
\]

\paragraph{Trades.}
We let $w_-$ and $c_-$ denote the pre-trade portfolio weights, \ie, the
portfolio weights before we carry out the trades to construct the portfolio
given by $w$ and $c$.
We will need these weights to incorporate transaction costs in the Markowitz formulation.
We refer to $z=w-w_-$, the current weights minus the previous ones,
as the (vector of) trades or the trade list.  Since $\ones^T w_- + c_-=1$, we have 
\[
c = c_- - \ones^T z,
\]
\ie, the post-trade cash weight is the pre-trade cash weight minus the total
weight of the trades.

The quantity
\[
T = \sum_{i=1}^n |z_i| = \| z\|_1
\]
is the \emph{turnover}.  (Here too several other different but closely related 
definitions are also used.)

\subsection*{Intermezzo}
We can already glimpse the Markowitz summit in the distance.

Our goal in portfolio construction is to choose the weights $w$ such that the
expected return after costs is maximized,
subject to constraints on the risk and the trading activity.
\[
\arg \max r^T w + r^\mathrm{rf} c - \mathrm{costs}(w, w_0)
\]
subject to
\[
    \mathrm{risk} \leq \mathrm{risk budget} \\
    \mathrm{trading} \leq \mathrm{trading budget} \\
    \mathrm{Constraints on the weights}
\]
Already in this naive formulation we may run into problems.
The risk may spike up and the trading required to get the risk below its threshold
may be too large.

%Constraints on the weights may help to reduce or boost
%the correlation to the underlying market.
%Trading a long only and diversified portfolio of S \& P stocks
%will result in a portfolio with an unavoidable high correlation
%to the index.

%In the next sections we make our formulation more precise and introduce
%models for risk and trading.


\subsection{Soft and hard constraints}
We classify constraints into soft and hard.
Hard constraints simply must be satisfied.
For example, if the portfolio must be long-only, then $w\geq 0$
is a hard constraint.

Using several hard constraints will often yield infeasible problems, e.g. a dramatic
change in our risk estimation may require trading beyond the limit for its activity.
There is an entire industry solving problems such as
\[
   \arg \max r^T w + r^\mathrm{rf} c - \lambda \mathrm{risk}(w)
\]
subject to
\[
    \mathrm{trading} \leq \mathrm{trading budget} \\
    \mathrm{Constraints on the weights}
\]
XXX: Thomas needs to learn LaTeX again

There people perform backtests to identify suitable values for the risk aversion
parameter $\lambda$ to end up with a portfolio somewhat close to the desired risk
level. We will not pursue this indirect approach here.

%We replace hard constraints with a combination of a soft objective with a far less restrictive hard constraint.
We will instead use a standard form for a \emph{soft constraints} which is a combination
of an objective with a far less restrictive hard constraint. It requires
to specify a target value for the objective, a limit value, and a priority.

%which specify for the objective
%a target value, a limit value, and a priority.
Consider an objective $f\in \reals$ such as risk for which
smaller values are better than larger values, \ie, we wish to minimize it.
(We will see below how the ideas can be transcribed to
objectives that we wish to maximize or remain in an interval.)

To capture our wishes and requirements for $f$, we give three
numbers, described below.
\BIT
\item The \emph{target value} $f^\mathrm{tar}$ is one that would satisfy us;
no effort should be expended
to push the value of $f$ below $f^\mathrm{tar}$.
\item The \emph{limit value} $f^\mathrm{lim}$ is the maximum value we can tolerate, \ie,
$f \leq f^\mathrm{lim}$ is a (hard) constraint.
\item The \emph{priority} $\omega>0$ tells us how much to weight values of $f$
between $f^\mathrm{tar}$ and $f^\mathrm{lim}$ when combining with other
objective terms.
\EIT
As an example, suppose that $f$ represents the annualized risk of a portfolio.
We might set a target value of 5\%, meaning that if the risk comes in under
this value, we don't care. We might set a limit of 15\%, which means we should
never construct a portfolio with a risk exceeding 15\%.

For an objective $f$ we wish to minimize we have $f^\mathrm{tar} < f^\mathrm{lim}$.
We associate with it a cost term
\[
\varphi = \omega \frac{(f-f^\mathrm{tar})_+}{f^\mathrm{lim}-f^\mathrm{tar}},
\]
and the (hard) constraint $f \leq f^\mathrm{lim}$.
The cost term is the priority $\omega$ multiplied by the fraction
of the way $f$ is between the target and the limit.
The cost term varies between $0$ and $\omega$.
The smaller $\varphi$ is, the happier we are.

For an objective $f$ that we wish to maximize we create the cost term
\[
\varphi = \omega \frac{(f^\mathrm{tar}-f)_+}{f^\mathrm{tar}-f^\mathrm{lim}},
\]
where in this case $f^\mathrm{tar} > f^\mathrm{lim}$, and hard constraint
$f \geq f^\mathrm{lim}$.
Here too $\varphi$ lies between $0$ and $\omega$,
and the smaller $\varphi$ is, the happier we are.
As an example, suppose $f$ represents annualized expected return.
Here we might set a limit
(in this case, minimum acceptable value) of $2\%$, and a target of $15\%$.
This means we will not accept any portfolio with expected return less than 2\%,
and will be fully satisfied with any expected return exceeding $15\%$.

Introducing soft constraints for both risk and trading activity the
risk of an infeasible problem is dramatically reduced but still a theoretical possibility.

XXX: Warning if $\omega$ is chosen too small the solver may ignore the target value
as exceeding it by an amount $c$ will result in expected return $r > c$.

XXX: Language. RELU function?

\subsection{Portfolio construction}
\paragraph{Portfolio construction problem.}
Our goal in portfolio construction is to choose $w$, which indirectly gives us 
the trades via $z=w-w_-$, and cash weight via $c=1-\ones^T w$.

\paragraph{What we know.}
When we choose $w$ we do not know the return $r$, but we have a 
statistical model of it, based on a forecast.  
This also holds for several other quantities, such as the
trading volume and the bid-ask spread over the trading period,
both of which affect the transaction cost, described below.

\paragraph{Objectives and constraints.}
Our goals and requirements for portfolio construction can be divided into
two groups.
\BIT
\item \emph{Holding objectives and constraints}
depend on the asset and cash weights $w$ and $c$.
They include our (forecast) return and risk, and
limits on positions and exposures.
\item \emph{Trading objectives and constraints}
depend on the trades $z$.
They include transaction costs and turnover limits.
\EIT

%\paragraph{Soft and hard constraints.}
%We classify constraints into soft and hard.  Hard constraints simply must be
%satisfied.  For example, if the portfolio must be long-only, then $w\geq 0$
%is a hard constraint.

%Soft constraints or soft goals are associated with an objective quantity
%that we wish to be large or small, or in some range.
%An example might be a constraint on exposure to some
%sector, which we would like to be between $\pm 3\%$, but if needed, we will
%accept some violation.


\subsection{Target-limit-priority specifications}
We will use a standard form for soft constraints, which specify for the objective
a target value, a limit value, and a priority.
Consider an objective $f\in \reals$ such as risk for which
smaller values are better than larger values, \ie, we wish to minimize it.
(We will see below how the ideas can be transcribed to
objectives that we wish to maximize or remain in an interval.)

To capture our wishes and requirements for $f$, we give three
numbers, described below. 
\BIT
\item The \emph{target value} $f^\mathrm{tar}$ is one that would satisfy us;
no effort should be expended
to push the value of $f$ below $f^\mathrm{tar}$.
\item The \emph{limit value} $f^\mathrm{lim}$ is the maximum value we can tolerate, \ie,
$f \leq f^\mathrm{lim}$ is a (hard) constraint.
\item The \emph{priority} $\omega>0$ tells us how much to weight values of $f$
between $f^\mathrm{tar}$ and $f^\mathrm{lim}$ when combining with other
objective terms.
\EIT
As an example, suppose that $f$ represents the annualized risk of a portfolio.
We might set a target value of 5\%, meaning that if the risk comes in under 
this value, we don't care.  We might set a limit of 15\%, which means we should 
never construct a portfolio with a risk exceeding 15\%.

For an objective $f$ we wish to minimize we have $f^\mathrm{tar} < f^\mathrm{lim}$.
We associate with it a cost term
\[
\varphi = \omega \frac{(f-f^\mathrm{tar})_+}{f^\mathrm{lim}-f^\mathrm{tar}},
\]
and the (hard) constraint $f \leq f^\mathrm{lim}$.
The cost term is the priority $\omega$ multiplied by the fraction 
of the way $f$ is between the target and the limit.
The cost term varies between $0$ and $\omega$.
The smaller $\varphi$ is, the happier we are.

For an objective $f$ that we wish to maximize we create the cost term
\[
\varphi = \omega \frac{(f^\mathrm{tar}-f)_+}{f^\mathrm{tar}-f^\mathrm{lim}},
\]
where in this case $f^\mathrm{tar} > f^\mathrm{lim}$, and hard constraint
$f \geq f^\mathrm{lim}$.
Here too $\varphi$ lies between $0$ and $\omega$,
and the smaller $\varphi$ is, the happier we are.
As an example, suppose $f$ represents annualized expected return.  
Here we might set a limit 
(in this case, minimum acceptable value) of $2\%$, and a target of $15\%$.
This means we will not accept any portfolio with expected return less than 2\%,
and will be fully satisfied with any expected return exceeding $15\%$.

Finally, consider a case where we wish $f$ to lie in the range XXX.
XXX break into two terms, one for the lower limit and one for the upper limit.
Special case: a target value.

As an example consider our portfolio exposure to the market, which we wish to be
zero, or at least, small.  For such a soft constraint we have cost term
$\omega |f|$, and hard constraint $|f|\leq f^\mathrm{max}$.

Our overall problem will have
$K$ different soft constraints, labeled $k=1, \ldots, K$,
with objective values $f_1, \ldots, f_K$.
For each of these we specify target values $f^\mathrm{tar}_k$, 
limit values $f^\mathrm{lim}_k$, and priorities $\omega_k$.
The individual cost terms are summed into an overall objective as
\[
\varphi = \sum_{k=1}^K \varphi_k.
\]
We will minimize $\varphi$ subject to the constraints to find our portoflio.

\subsection{Portfolio construction inputs}
Our portfolio construction method requires several data to run.  These 
can be classified into forecasts of quantities that are not known at construction time,
and parameters that specify our objectives and constraints.

Forecasts:
\BIT
\item return forecast $\alpha$
\item risk model $\Sigma$, typically given as factor model
\item bid-ask spreads,
volatility, and trading volume (needed to specify transaction costs)
\EIT

Parameters specify the hard constraints, as well as the parameters
needed to specify the soft constraints, \ie, target, limit, and priority.
\BIT
\item target risk level
\item limits on weights and factor exposures
\item limits on trades, \eg, turnover
\item limits on liquidation cost
\EIT


\paragraph{Return.}

$\alpha^T w + r^\mathrm{rf} c$

Here $\alpha$ is an estimate of the return of the assets.

Can add an uncertainty band for the return and do worst-case return.

\paragraph{Risk.}
XXX be sure to mention diffference between using adjusted return 
and our method which uses a soft constraint for risk.

$\Sigma$ is an estimate of the covariance of $r$.

The risk is $w^T\Sigma w$.  Its squareroot is the volatility of portfolio return,
and both more interpretable and better behaved numerically.  We will work with
that.  
\[
\sigma = \sqrt{w^T \Sigma w} = \|\Sigma^{1/2}w \|_2,
\]
where $\Sigma^{1/2}$ is a squareroot of $\Sigma$, \eg, its Cholesky factor.

\paragraph{Factor risk model.}
We will use a factor form for $\Sigma$.  Define $F$, $\Sigma^\mathrm{f}$, $D$.
Let $f$ denote the factor return.

Risk can be expressed as 
\[
\sigma = \left\| (\| (\Sigma^\mathrm{rf})^{1/2} f \|_2, \|D^{1/2}w\|_2 ) \right\|_2
\]

Risk uncertainty; see Kahn paper.

\paragraph{Holding costs.}

\paragraph{Trading costs.}

\clearpage
Markowitz \cite{markowitz1952portfolio} understood his contribution as the second step in a process
where the first step is to estimate the expected returns.
A lot of practitioners leave money on the table by investing all their efforts into estimating
the expected returns but rely on rather naive methods to construct the portfolio.
Needless to say that the construction of expected returns is often done using
a convex optimization problem combining several standard estimators per asset.

XXX This method is known as Maximum Diversification and has been introduced
by Choueifaty, Coignard and Reynier\cite{choueifaty2011toward}.

\subsection{The expected risk}

The expected risk is a non-negative measure to quantify the risk of a portfolio,
see \cite{delbaen2002coherent} for a deeper discussion.
The most common measure is the expected variance of the portfolio, \eg
\[
\sigma^2 = \Expect[(w^T r - \mu)^2] = \Expect[(w^T r - w^T \mu)^2] = w^T \Sigma w
\]
where
\[
    \Sigma_{i,j} = \Expect[(r_i - \mu_i)(r_j - \mu_j)^T]
\]
are the elements of the variance-covariance matrix $\Sigma$.

\paragraph{Soft constraints.}
as described in~\cite[\S4.2]{BoydKahnMultiPeriod}.


\subsection{Costs and trade impact}

An accurate reflection of expected costs and trade impact is most crucial in particular
for large portfolios or portfolios with high turnover. Typically, the costs are
usually cited as a percentage of the traded value. More relevant for our purposes
is the trade impact, which is the change in price due to the trade. This is a direct
consequence of our market order eating into the order book. The trade impact is
usually modeled as a function of the trade size, the volatility of the asset and
the relative volume of the trade.

Note that a linear trade impact model should be avoided as it is not realistic
but has no negative effect on a growing portfolio size. A quadratic model is usually
to harsh on large orders and is not realistic either. We go for the more
releastic model in spirit of Almgren and Chriss~\cite{almgren2000optimal} and scale with $p=3/2$
(There is a lot in Grinold on Kahn on the sqrt rule?)

We also offer users the flexibility to implement their own models for costs and
trade impact via inheritance as long as they are convex.

For the experiments in this paper we use the following model for the trade impact
...

\subsection{Holding objectives and constraints}
We want to express preferences and limits for the constructed portfolio which
depend on the weights $w$ and the cash weight $c$. A basic holding constraint
is derived from our definition of the portfolio weights $w$ and the cash weight,
requiring that the sum of asset weights and the cash weight is equal to one,
\ie, $\ones^T w + c = 1$. This constraint has to be satisfied for any portfolio
and is therefore a hard constraint. Most other constraints can, depending on the
application, be expressed either as pure preferences in the objective,
soft constraints, or hard constraints. We will therefore describe holding
objectives and constraints generally as functions of $w$ and $c$, denoted by
$\phi^{\text{hold}}(w, c)$. We give some examples of common holding objectives
and constraints below.

\paragraph{Expected return.}
TODO(Philipp): I can add highly condensed description of each paragraph below
if we like the format

\paragraph{Risk.}

\paragraph{Lower and upper bounds.}

\paragraph{Leverage limit.}

\paragraph{Group bounds.}

\paragraph{Factor exposures.}

\paragraph{Other constraints.}
There are many other constraints that may be useful in practice.
Generally, when $\phi^{\text{hold}}(w, c)$ is a convex function
To give some examples, we may want to limit the concentration of the portfolio,
\eg, by requiring that the sum of the largest $k$ weights is smaller than a
certain value. Other examples include tracking error to a benchmark,

\subsection{Trading objectives and constraints}
Here we define $\phi^{\text{tc}}(z)$.

\subsection{Linear constraints and factor models}

Factor risk models use the projection of the asset returns into a lower
dimensional subspace, \eg each asset is the linear combination of $k$ factors.
\[
r_i = \sum_{j=1}^k x_j \beta_{ji} + \epsilon_i
\]
The factor time series are $x_1, \ldots, x_k$ with $k < n$
The loadings are the coefficients $\beta_{ij}$.
The residual returns $\epsilon_i$ are assumed to be uncorrelated with the factors.
\[
  R = X \beta^T + \epsilon
\]
Any position $w$ in weight space projects to a position $f = \beta^T w$ in factor space.
Similar to the portfolio weights, we can use linear constraints to restrict
the position in factor space, \eg we can limit the exposure
to individual factors or groups of factors.

The variance for a position $w$ is the sum of the variance of the
systematic returns explained by the factors and the variance of the idiosyncratic returns.

\[
\var(r) = \var(X \beta^T w) + \var(\epsilon w)
\]
We assume the residual returns are uncorrelated and hence
\[
\var(r) = y^T \Sigma_f y + \sum_i w_i^2 \var(\epsilon_i)
\]
where $\Sigma_f$ is the covariance matrix of the factors and $\var(\epsilon_i)$
is the variance of the idiosyncratic returns.

There are two popular approaches to construct factor risk models in practice.
The more fundamental approach is the user providing the factor covariance matrix $\Sigma_f$,
the loadings $\beta$ and the volatilities of the idiosyncratic returns $\epsilon_i$.
Often such data is provided by external parties, e.g. Barra or Axioma.

In the second variant the user provides the factor time series $x_1, \ldots, x_k$, usually obtained from
a principal component analysis (PCA) of the asset returns. The loadings $\beta$ are computed
via a linear regression of the asset returns on the factor time series.
The volatilities of the idiosyncratic returns $\epsilon_i$ are computed as the standard deviation
of the residuals of the linear regression.
The factor covariance matrix $\Sigma_f$ may even be diagonal in this case as the factors are orthogonal.


%The reason is that the expected return and the expected risk are not on the same scale.


%Max expected return with risk constraint
%Make it an inequality constraint, not an equality constraint, to enforce convexity
%Do not use Max Expected return - lambda * risk

%Use weights (fraction of capital)

%Variation:
%Max a (w - w0)
%where w0 is given. outperform some index or benckmark



%Lower and upper bounds on w (\eg, long only)

%Linear constraint like sum w = 1 (fully invested)

%or sum w = 0 (to be cash neutral).

%Establish factors as F*w=f     (F is the factor matrix, it is wide)

%upper and lower constraints on f

%control exposure to countries, sectors, etc

%Constraints on leverage

%sum abs(w) <= C

%Keep an inequality for convexity of the feasible domain

Penalties in objective for trading impact and holding costs.

Loads of research on trade impact

%Numerous ideas to estimate risk, see cvxrisk

Emphasize that this framework is rather versatile although it does not cover all 
thinkable applications.
It covers everything we have seen in practice

\subsection{The general problem}
Below we describe the general problem that we want to solve. It includes
many common portfolio construction problems as special cases, some of which
we will point out explicitly.

% Still missing: factors, robust risk/return
\begin{equation}
\begin{array}{ll}
\mbox{maximize}   &  w^T\mu - \lambda (\sigma - \sigma^{\text{target}})_+ -
\gamma^{\text{hold}}\phi^{\text{hold}}(w) - \gamma^{\text{tc}}\phi^{\text{tc}}(w)\\
\mbox{subject to} &  \ones^T w = 1,\\
                  &  \|w_{1:n}\|_1 \leq L_{\max},\\
                  &  \|\Sigma^{1/2} w\|_2 \leq \sigma,\\
                  &  \sigma \leq \sigma^{\text{limit}},\\
                  &  w \in \mathcal{W},\\
\end{array}
\label{eq:general}
\end{equation}

where the optimization variable is the portfolio weights $w \in \R^n$, and
$\sigma \in \reals_+$ is an auxiliary variable. The parameters of the problem
are the expected returns $\mu \in \R^n$, the covariance matrix $\Sigma \in \symm^n_+$,
the target volatility $\sigma^{\text{target}} \in \reals_+$, the maximum
volatility $\sigma^{\text{limit}} \in \reals_+$, the maximum leverage
$L_{\max} \in \reals_+$, as well as the priorities $\lambda \in \reals_+$,
$\gamma^{\text{hold}} \in \reals_+$, and $\gamma^{\text{tc}} \in \reals_+$ of
the risk, holding cost, and transaction cost terms, respectively. Additional
constraints on the weights, such as asset specific upper and lower bounds, as
well as group constraints, are encoded in the convex set $\mathcal{W}$.
We also allow some weight constraints to be soft, in which case we replace
the corresponding constraints with penalties in the objective function.
The functions $\phi^{\text{hold}}$ and $\phi^{\text{tc}}$ are convex functions
that model the holding costs and transaction costs.


\section{Some crazy ideas}

Here's what a portfolio construction method is supposed to do.
It should handle constraints (\eg, long only), and trade off risk,
return, transaction cost, and holding cost (typically shorting cost).
It will evidently have several hyperparameters, that allow the user
to choose particular trade-offs among these objectives.
These can be chosen by hand, or by hyperparameter search, as described below.

First some general ideas about using an optimizer in a policy.
\BIT
\item You do not really want to solve the optimization problem.  It is only a surrogate
for what you want to do, which is create a trading policy that does well, doesn't trade too much or 
expose you to excessive risk, all on unseen (out of sample) data.
Same story in every field that uses optimization, \eg, machine learning, control, prediction.
\item This means that don't need to worry about the terms or constraints in the problem being `accurate'.
Typical idea is to start with a form that is roughly what you want (say, quadratic risk), and scale 
it up and down with a hyper-parameter.
\EIT

\paragraph{Safety fencing.}  You put in constraints that you don’t expect to be tight;
they are there just to keep us out of trouble.  Each of these constraints comes
with an alarm; if they ever are tight, someone gets beeped/paged!  Examples:
limits on individual positions, limits on factor exposures (even if the factors
are already in the risk model). Could even add diversification constraints such
as no more than 20\% of the portfolio can be concentrated in 5 names or fewer.

(We should be able to do this in CVXPY, because it’s a general thing.)

\paragraph{Regularization of data.}  Want to be robust to errors in estimating $\mu$ and
$\Sigma$.  Ideally, we should be able to handle a nominal $\Sigma$ that is
singular. Assertions that throw and exception if some data is way outside a
reasonable range (eg, crazy returns, covariances, volumes, bid-ask spreads, …)
Much written about how to estimate covariances, means for Markowitz.  Black Litterman.

Winsorize.  Use robust estimator losses like Huber.

\paragraph{Cleanup phase.}   Minimum nonzero trade/holdings.  Max number of
trades or names.  Integer lots numbers.  Fix cost for transactions.
No need for silly global mixed integer -- two passes of convex are just fine.
Say why!

\paragraph{Fast backtest.}  We can solve realistic portfolio optimization problems in 10
or 10s of ms.  So 1 4 year daily backtest costs around 10s, single thread.  Now
run 64 backtests in parallel on a machine with 64 hyper threads, and go to
lunch for one hour.   Congratulations; you just did 23000 backtests.   That’s a
lot of hyper-parameter exploring you can (and should!) do. These could be done
without some constraints such as the cleanup.  For the few that are
interesting, you can rerun them in high accuracy mode.   Another option is to
differentiate through the optimizer to get some kind of gradient w.r.t.
hyperparameters.   We can also mention parameters in CVXPY and also CVXPYgen.

It would be fun to have with this a reference implementation (in CVPXY, of
course), that actually works really well.  I don’t see why we couldn’t do that.

I’ve never looked at commercial portfolio construction tools like Axioma.  But
I know BLK’s one, and it’s terrible and very slow.

\paragraph{Backtests.}
Super important: adjusting hyper-parameters via backtest.




\paragraph{What if experiments.}
Should also do peridioc what-if experiments, where you go back 4 months, say,
and change some hyper-paramters and see where you'd be today had you used those values.

\paragraph{Kalman filters and linear quadratic control.}
One analogy we can make is to the Kalman filter or linear quadratic control.   
These are used EVERYWHERE
and very successfully.  If you read a theory book about it and implement the
algorithm it will never work.  But with just a few tricks (not too far from the
ones we are going to describe), they work super well in practice. No one whines
about “Kalman filters don’t work unless you get the noise covariance right”.
(In any case, that's completely false.  The issue is not getting an `accurate' noise or 
process covariance; the issue is getting an \emph{appropriate} one.
Instead, people just know what to do.   Markowitz seems very similar to us. You
have to know some tricks, if you want to make it work in practice.

\paragraph{Beware the shiny new method.}  Your colleagues/clients will tell you “no one
uses Markowitz anymore. Instead they use XX, YY, or ZZ methods”.  Our advice:
Do not believe them.  No, you do not need CVaR or a sampling based method (in
practice) or even, likely, a multi-period formulation.
Or the objective of the month like Sortino half downside risk, or
blah blah.   

Linear quadratic control flies all airplanes, quite safely and
well, thank you; in a similar way, Markowitz is just fine for portfolio
construction.

\paragraph{Let the optimizer do its job.}   One of its jobs is to
control risk; another is control turnover/t-cost.  Let it do it; don’t smooth your
signal or complain that a colleagues mean prediction `has too much turnover' or `too much risk'. 

\paragraph{Work with weights.}
In the optimizer, work with weights that sum to one, not shares or dollar values.  For
some purposes it doesn’t matter (\eg, leverage); for others (3/2 power transaction
costs) it does.  Adjust things in the solver to work with weights by passing in
the total portfolio value.  Outside the solver, feel free to use common and
sensible units like shares or dollar values.  Just not in the code of the
optimizer.

\section{Input data}

\paragraph{Covariance estimation.}  Can be fancy, but very simple ones work just as well, 
and with much less babysitting.

\paragraph{Mean estimation.}  This is the secret sauce. Mention some simple ones?

\paragraph{Volatility estimation.}  (Used for t-cost.) See covariance estimation.

\paragraph{Volume estimation.}  (Used for t-cost.)  Average trailing is just fine.

\paragraph{Bid-ask spread cost.}  (Used for t-cost.) Use true one or just make it up.

\section{Technology}

To support our ideas we have implemented a dedicated Python package
for portfolio construction. The cvxmarkowitz package is built on top of the cvxpy package \cite{cvxpy}.
Our core design principle has been to hide the complexity of the subject behind
a rather intuitive interface. The package is open source and available on GitHub.

In particular, we take advantage of the new features of cvxpy 1.4?  enabling the smart
caching of problems. This allows us to solve problems as discussed here in a few milliseconds.
We discuss the technical details of the implementation in an alternative paper.

We shall emphasize that the package provides a degree of flexibility not discussed in this paper,
e.g., if you prefer a customized risk model, or a different trade impact model, etc. you
could integrate all such requirements using inheritance from the base classes provided by the package.

\section{Back-testing and parameter selection}

A backtest is a simulation of a trading strategy using historical data.
We loop through time and at each time step we solve the portfolio optimization problem
using historic or artificial price data.
This loop is independent of the allocation
scheme for the portfolio.
We can use the same loop for a Markowitz portfolio,
but we could also reuse it for evaluating the performance of a monkey throwing darts
on a board with names of stocks.

\begin{verbatim}
import numpy as np
import pandas as pd

from cvx.simulator.builder import builder

prices = pd.read_csv("price.csv", index_col=0, parse_dates=True, header=0).ffill()
b = builder(prices=prices, initial_cash=1e6)

for t, state in b:
    # pick two assets at random
    pair = np.random.choice(b.assets, 2, replace=False)
    # compute the pair
    stocks = pd.Series(index=b.assets, data=0.0)
    stocks[pair] = [state.nav, -state.nav] / state.prices[pair].values

    # update the position, 10% of the nav long, 10% short
    b[t[-1]] = 0.1 * stocks

# build the portfolio
portfolio = b.build()
\end{verbatim}
Explain the source code? Better environment for source code?

The constructed portfolio object opens the door to further analytics wide.
We are using the cvxsimulator package here. CITATION?


\input{sections/conclusions.tex}


\clearpage
\bibliography{refs}

\clearpage
\appendix

\end{document}
